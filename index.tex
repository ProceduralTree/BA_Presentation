%& /home/jon/.config/emacs/.local/cache/org/persist/20/054288-3473-4ffe-8933-01a9c5fe54d0-59a0bd2da087ae89cfcca34af1a70371
% Intended LaTeX compiler: pdflatex
\documentclass[presentation]{beamer}
\usepackage[utf8]{inputenc}
\usepackage[T1]{fontenc}
\usepackage{hyperref}
\usepackage{longtable}
\usepackage{xcolor}
\usepackage{float}
\usepackage{amssymb}
\usepackage{amsthm}
\usepackage{amsmath}

%% ox-latex features:
%   !announce-start, !guess-pollyglossia, !guess-babel, !guess-inputenc, maths,
%   image, svg, !announce-end.

\usepackage{amsmath}
\usepackage{amssymb}

\usepackage{graphicx}

\usepackage[inkscapelatex=false]{svg}

%% end ox-latex features


% end precompiled preamble
\ifcsname endofdump\endcsname\endofdump\fi

\usetheme{default}
\author{Jonathan Ulmer}
\date{\today}
\title{Numerical methods}
\subtitle{on the Cahn-Hilliard Equation}
\hypersetup{
 pdfauthor={Jonathan Ulmer},
 pdftitle={Numerical methods},
 pdfkeywords={},
 pdfsubject={},
 pdfcreator={Emacs 29.3 (Org mode 9.7-pre)}, 
 pdflang={English}}
\begin{document}

\maketitle
\begin{frame}[label={sec:orgd620e75}]{Table of Contents}
\end{frame}
\begin{frame}[label={sec:org1caa142}]{Introduction}
\end{frame}

\begin{frame}[label={sec:orga77819f}]{The Cahn-Hilliard equation}
\begin{center}
\begin{tabular}{ll}
\(M(\phi): [-1,1] \to \mathbb{R}^+\) & Mobility coefficient\\
\(W(\phi): [-1,1] \to \mathbb{R}^+\) & Double well potential\\
\(\varepsilon > 0 \in \mathbb{R}\) & Interface coefficient\\
\(\phi : \Omega \times (0,T) \to \mathbb{R}^d\) & Phase-field \alert{variable}\\
\(\mu : \Omega \times (0,T) \to \mathbb{R}^d\) & Chemical potential \alert{variable}\\
\end{tabular}
\end{center}
\begin{block}{Cahn-Hilliard Equation:}
\begin{equation}
\label{eq:CH}
\begin{aligned}
\partial_{t}\phi(x,t) &=  \nabla \cdot(M(\phi)\nabla\mu), \\
\mu &= - \varepsilon^2 \Delta\phi  + W'(\phi),
\end{aligned}
\end{equation}
\begin{itemize}
\item phase field equation for two phase flow
\item diffuse interface equation
\item gives position of phases
\end{itemize}

\begin{itemize}
\item constant mobility \(M(\phi) \equiv 1\)
\item polynomial potential \(W(\phi) = \frac{1}{4} \phi^2(1-\phi^2)\)
\item 0 Neumann boundary conditions
\end{itemize}
\begin{equation}
\label{eq:boundary-conditions}
\begin{aligned}
\nabla\mu \cdot \mathbf{n} &= 0 & \text{on} \, \partial\Omega &\times (0,T),\\
\partial_n\phi &= 0 & \text{on} \, \partial\Omega &\times (0,T),
\end{aligned}
\end{equation}
\end{block}
\begin{block}{Properties}
\begin{itemize}
\item Mass conservative
\item Total energy decreases
\item expensive
\end{itemize}
\end{block}
\end{frame}
\begin{frame}[label={sec:orgb21f9df}]{Baseline solver}
\begin{itemize}
\item Implicit in time
\item discretized on NxN grid
\item uses multi-grid scheme
\end{itemize}
\begin{center}
\begin{tabular}{ll}
\(b\) & collects all terms not dependant on \(\phi_{ij}^{n+1}\)\\
\(DL\) & Jacobian of \(L\)\\
\(L\) & Implicit terms of the discrete CH equation\\
\(\Omega_d\) & discrete version of the computational domain \(\Omega\)\\
\end{tabular}
\end{center}
\begin{itemize}
\item solves equation of type
\end{itemize}
\begin{equation}
DL \cdot
\begin{pmatrix}
\phi^{n+1}_{ij} \\
\mu^{n+\frac{1}{2}}_{ij}
\end{pmatrix}
= b
\end{equation}
\begin{itemize}
\item with Gauss Seidel iteration
\item for every point \((i,j) \in \Omega_d\)
\item on two grid scales
\item multiple times per sub-iteration
\end{itemize}
\end{frame}
\begin{frame}[label={sec:orgc31f08d}]{Relaxation}
\begin{block}{Relaxed Cahn Hilliard equation}
\begin{equation}
\label{eq:relaxed-cahn-hilliard}
\begin{aligned}
\partial_t \phi^\alpha  &= \Delta \mu \,,\\
\mu &= \varepsilon ^2 \alpha(c^\alpha - \phi^\alpha) + W'(\phi) .
\end{aligned}
\end{equation}
where  \(\alpha < 0\)  is a relaxation parameter
\end{block}
\begin{block}{Additional elliptical system}
\begin{align}
\label{eq:elliptical-equation}
- \Delta c^\alpha  + \alpha c^a &= \alpha \phi ^\alpha,
\end{align}
\begin{itemize}
\item requires solving an additional equation for \(c\)
\item two dependant equations
\item two one dimensional second order equations
\item solved similar to the baseline equation
\end{itemize}
\end{block}
\begin{block}{Implementation}
\begin{itemize}
\item implicit in \(c\)
\item solves both equations in tandem
\item resolving \(c\) during each sub-iteration required
\end{itemize}
\end{block}
\end{frame}
\begin{frame}[label={sec:org596cfe3}]{Numerical Experiments}
\begin{block}{Energy}
\begin{center}
\includesvg[width=.9\linewidth]{images/energy_balance}
\label{fig:energy-balance}
\end{center}

\begin{center}
\includesvg[width=.9\linewidth]{images/relaxed-energy-balance}
\label{fig:relaxed-energy-balance}
\end{center}
\end{block}
\begin{block}{mass conservation}
\begin{center}
\includesvg[width=.9\linewidth]{images/relaxed-mass-balance}
\label{fig:relaxed-mass-balance}
\end{center}
\end{block}
\begin{block}{Sub iteration}
\begin{center}
\includesvg[width=.9\linewidth]{images/relaxed-convergence}
\label{fig:relaxed-convergence}
\end{center}
\end{block}
\begin{block}{time}
\begin{center}
\includesvg[width=.9\linewidth]{images/relaxed-time-stability}
\label{fig:relaxed-stability-in-time}
\end{center}
\end{block}
\end{frame}
\begin{frame}[label={sec:org9283679}]{Comparison}
\url{images/relaxed-comparison.gif}

\url{images/relaxed-anim.gif}

\url{images/iteration.gif}
\end{frame}
\begin{frame}[label={sec:org845d90f}]{Conclusion}
\end{frame}
\end{document}
