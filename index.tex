%& /home/jon/.config/emacs/.local/cache/org/persist/62/6a3bf8-22bf-43a5-a8dc-a4fb528c0299-cc1ee15f43b9835e6540b5eebedd3f2c
% Intended LaTeX compiler: pdflatex
\documentclass[11pt]{article}
\usepackage[utf8]{inputenc}
\usepackage[T1]{fontenc}
\usepackage{hyperref}
\usepackage{longtable}
\usepackage{xcolor}
\usepackage{float}
\usepackage{amssymb}
\usepackage{amsthm}
\usepackage{amsmath}

%% ox-latex features:
%   !announce-start, !guess-pollyglossia, !guess-babel, !guess-inputenc, caption,
%   maths, image, svg, !announce-end.

\usepackage{capt-of}

\usepackage{amsmath}
\usepackage{amssymb}

\usepackage{graphicx}

\usepackage[inkscapelatex=false]{svg}

%% end ox-latex features


% end precompiled preamble
\ifcsname endofdump\endcsname\endofdump\fi

\author{Jonathan Ulmer}
\date{\today}
\title{Numerical methods\\\medskip
\large on the Cahn-Hilliard Equation}
\hypersetup{
 pdfauthor={Jonathan Ulmer},
 pdftitle={Numerical methods},
 pdfkeywords={},
 pdfsubject={},
 pdfcreator={Emacs 29.3 (Org mode 9.7-pre)}, 
 pdflang={English}}
\begin{document}

\maketitle
\section*{Table of Contents}
\label{sec:org8a33c53}
\section*{The Cahn-Hilliard equation}
\label{sec:org0cb516e}
\subsection*{Variables}
\label{sec:org4f9ea90}
\begin{center}
\begin{tabular}{ll}
\(M(\phi): [-1,1] \to \mathbb{R}^+\) & Mobility coefficient\\
\(W(\phi): [-1,1] \to \mathbb{R}^+\) & Double well potential\\
\(\varepsilon > 0 \in \mathbb{R}\) & Interface coefficient\\
\(\phi : \Omega \times (0,T) \to \mathbb{R}^d\) & Phase-field \textbf{variable}\\
\(\mu : \Omega \times (0,T) \to \mathbb{R}^d\) & Chemical potential \textbf{variable}\\
\end{tabular}
\end{center}
\begin{figure}[htbp]
\centering
\includesvg[width=.9\linewidth]{images/domain}
\caption{\label{fig:domain}Computational Domain \(\Omega\)}
\end{figure}
\begin{figure}[htbp]
\centering
\includesvg[width=.9\linewidth]{images/double-well}
\caption{\label{fig:double-well}Double well potential \(W(\phi)\)}
\end{figure}
\subsection*{Cahn-Hilliard Equation:}
\label{sec:org1046473}
The aim of the CH equation is to find solutions \(\phi(\vec{x} , t) , \mu(\vec{x} , t): \Omega \times (0,T) \to \mathbb{R}\) such that they satisfy
\begin{equation}
\label{eq:initial-value-problem}
\begin{aligned}
\partial_{t}\phi(x,t) &=  \nabla \cdot(M(\phi)\nabla\mu),\\
\mu &= - \varepsilon^2 \Delta\phi  + W'(\phi), & \text{in} \, \Omega &\times (0,T),\\
-\nabla\mu \cdot \mathbf{n} &= 0\\
\nabla\phi \cdot \mathbf{n} &= 0 & \text{on} \, \partial\Omega &\times (0,T), \\
\phi(x,0) &= \phi^0(x) \,, & \text{in} \, \Omega &
\end{aligned}
\end{equation}

\begin{itemize}
\item constant mobility \(M(\phi) \equiv 1\)
\item polynomial potential \(W(\phi) = \frac{1}{4} \phi^2(1-\phi^2)\)
\item no flow Neumann boundary conditions
\end{itemize}
\subsection*{Properties}
\label{sec:org5836bbb}
\begin{itemize}
\item Mass conservative
\item Total energy decreases
\item phase field equation for two phase flow
\begin{itemize}
\item diffuse interface equation
\item gives position of phases
\end{itemize}
\item 4th order (2 dimensional second order)
\begin{itemize}
\item expensive
\end{itemize}
\end{itemize}
\section*{Baseline solver}
\label{sec:org1ce16b2}
\begin{itemize}
\item Implicit in time
\item discretized on NxN grid
\item uses multi-grid scheme
\end{itemize}
\subsection*{Discrete Method:}
\label{sec:org3019139}
\begin{equation}
\begin{aligned}
\partial_{t}\phi(x,t) &=  \nabla \cdot(M(\phi)\nabla\mu), \\
\mu &= - \varepsilon^2 \Delta\phi  + W'(\phi),
\end{aligned}
\end{equation}
\begin{itemize}
\item semi implicit in time
\item centered difference in space \autocite{SHIN20117441} .
\end{itemize}
\subsection*{Discrete Method:}
\label{sec:org5f43651}
\begin{equation}
\label{eq:discrete-cahn-hilliard}
\begin{aligned}
\frac{\color{RoyalBlue}{\phi_{ij}^{n+1}} - \color{Maroon}{\phi_{ij}^n}}{\Delta t}  &=  \color{RoyalBlue}{\nabla _d \cdot (G_{ij} \nabla_d \mu_{ij}^{n+\frac{1}{2}} )}  \,, \\
 \color{RoyalBlue}{\mu_{ij}^{n+\frac{1}{2}}} &= \color{RoyalBlue}{2\phi_{ij}^{n+1}} - \varepsilon^2  \color{RoyalBlue}{\nabla_d \cdot  (G_{ij} \nabla _d \phi_{ij}^{n+1} )} + \color{Maroon}{W'(\phi_{ij}^n) - 2\phi _{ij}^n} \,,
\end{aligned}
\end{equation}
\begin{itemize}
\item semi implicit in time
\item centered difference in space \autocite{SHIN20117441} .
\end{itemize}
\subsection*{Variables}
\label{sec:orgaddb669}
\begin{center}
\begin{tabular}{ll}
\(b\) & \(b = DL \cdot \left( \phi_{ij}^{n+1} , \mu_{ij}^{n+\frac{1}{2}} \right)^T - L \left(  \phi_{ij}^{n+1} , \mu_{ij}^{n+\frac{1}{2}}  \right)\)\\
\(DL\) & Jacobian of \(L\)\\
\(\color{RoyalBlue}{L}\) & Implicit terms of the discrete CH equation\\
\(\color{Maroon}{\left( \zeta_{ij}^n , \psi_{ij}^{n} \right)}\) & Explicit terms of the discrete CH equation\\
\(\Omega_d\) & discrete version of the computational domain \(\Omega\)\\
\end{tabular}
\end{center}
\subsection*{solver}
\label{sec:orgb4c20cd}
\begin{itemize}
\item solves equation of type
\end{itemize}
\begin{equation}
DL \cdot
\begin{pmatrix}
\phi^{n+1}_{ij} \\
\mu^{n+\frac{1}{2}}_{ij}
\end{pmatrix}
= b
\end{equation}
\begin{itemize}
\item with Gauss Seidel iteration
\item for every point \((i,j) \in \Omega_d\)
\item using a  two grid method
\item multiple times per sub-iteration
\item multiple sub-iterations per time-step
\end{itemize}
\section*{Relaxation}
\label{sec:org1f6ab6d}
\subsection*{Variables}
\label{sec:org20d239b}
\begin{center}
\begin{tabular}{ll}
\(\alpha > 0\) & relaxation parameter\\
\(c^{\alpha}:\Omega \times (0,T) \to \mathbb{R}^d\) & solution of an elliptical system\\
\end{tabular}
\end{center}
\subsection*{Relaxed Cahn Hilliard equation}
\label{sec:orgf65c34e}
\begin{equation}
\label{eq:relaxed-cahn-hilliard}
\begin{aligned}
\partial_{t}\phi^{\alpha}(x,t) &=  \Delta\mu, \\
\mu &= - \varepsilon^2 \Delta\phi^{\alpha}  + W'(\phi^{\alpha}),
\end{aligned}
\end{equation}
\begin{itemize}
\item simpler PDE
\item second order
\end{itemize}
\subsection*{Relaxed Cahn Hilliard equation}
\label{sec:orge441d95}
\begin{equation}
\label{eq:relaxed-cahn-hilliard}
\begin{aligned}
\partial_t \phi^\alpha(x,t)  &= \Delta \mu \,,\\
\mu &= \varepsilon ^2 \alpha(c^\alpha - \phi^\alpha) + W'(\phi^{\alpha}) .
\end{aligned}
\end{equation}
\begin{itemize}
\item simpler PDE
\item second order
\end{itemize}
\subsection*{Additional elliptical system}
\label{sec:orge10da20}
\begin{align}
\label{eq:elliptical-equation}
- \Delta c^\alpha  + \alpha c^a &= \alpha \phi ^\alpha,
\end{align}
\begin{itemize}
\item requires solving an additional equation for \(c\)
\item two dependant equations
\item two one dimensional second order equations
\item solved similar to the baseline equation
\end{itemize}
\subsection*{Implementation}
\label{sec:orgf602ff2}
\begin{itemize}
\item implicit in \(c\)
\item solves both equations in tandem
\item resolving \(c\) during each sub-iteration required
\end{itemize}
\subsection*{Exlicit and implicit solution of c}
\label{sec:org8066d72}
\begin{figure}[htbp]
\centering
\includesvg[width=.9\linewidth]{images/explicit-elips-smooth}
\caption{\label{fig:relaxed-smooth-eval}exlicit solution of c}
\end{figure}
\begin{figure}[htbp]
\centering
\includesvg[width=.9\linewidth]{images/alternating-elips-smooth}
\caption{\label{fig:alternating-elips-smooth}implicit solution of c by alternating solving \phhi and c}
\end{figure}
\subsection*{choice of \(\alpha\)}
\label{sec:orgeb497bd}
\begin{center}
\includesvg[width=.9\linewidth]{images/alpha-error}
\label{fig:alpha-error}
\end{center}
\section*{Numerical Experiments}
\label{sec:orga13f457}
\subsection*{Energy}
\label{sec:orgdcdbae9}
\begin{itemize}
\item The CH equation is related to the following energy functional
\[E^{\text{bulk}}[\phi] = \int_{\Omega} \frac{\varepsilon^2}{2} |\nabla \phi |^2 + W(\phi) \, dx  \]
\item The relaxed CH has the following related energy functional
\[E_{rel}(\phi^{\alpha} , c^\alpha) := \int_{\Omega}  \frac{1}{2}\varepsilon^2 \alpha (c^\alpha - \phi^{\alpha})^2 + W(x) \ d \operatorname{\mathbf{x}} \]
\item Total energy decreases for the CH equation and the relaxed CH equation
\[\frac{d}{dt}E[\phi(t)]  \stackrel{\partial_n\phi = 0}{=} - \int_{ \Omega } |\nabla \mu|^2 \ d \mathbf{x}, \qquad \forall t \in (0,T) \]
\item Relaxed CH equation should decrease both energy functionals
\end{itemize}
\begin{figure}[htbp]
\centering
\includesvg[width=.9\linewidth]{images/relaxed-energy-balance}
\caption{\label{fig:relaxed-energy-balance}Discrete Energy decrease in both solvers}
\end{figure}
\subsection*{mass conservation}
\label{sec:org8135f88}
\begin{equation}
\frac{d}{d t} \int_{\Omega} \phi ~\mathrm{d} \operatorname{\mathbf{x}} = 0
\end{equation}
\begin{itemize}
\item Discrete versions should satisfy
\end{itemize}
\begin{equation}
\sum_{i,j \in \Omega} \frac{\phi_{ij}^{n} - \phi_{ij}^{n+1}}{\Delta t} = 0
\end{equation}
\begin{figure}[htbp]
\centering
\includesvg[width=.9\linewidth]{images/mass_balance}
\caption{\label{fig:mass-balance}energy conservation for both solvers}
\end{figure}
\begin{figure}[htbp]
\centering
\includesvg[width=.9\linewidth]{images/relaxed-mass-balance}
\caption{\label{fig:relaxed-mass-balance}relaxed mass behavior}
\end{figure}
\subsection*{Sub iteration}
\label{sec:org4c9f08c}
\begin{figure}[htbp]
\centering
\includesvg[width=.9\linewidth]{images/relaxed-convergence}
\caption{\label{fig:relaxed-convergence}Behaviour of both solvers during sub-iterations}
\end{figure}
\subsection*{time}
\label{sec:orgec8065c}
\begin{figure}[htbp]
\centering
\includesvg[width=.9\linewidth]{images/relaxed-time-stability}
\caption{\label{fig:relaxed-stability-in-time}Behaviour of both solvers when varying time-step size}
\end{figure}
\subsection*{Direct comparison of the baseline solver with the relaxed solver}
\label{sec:org691f07d}

\url{images/relaxed-comparison.gif}
\section*{Conclusion}
\label{sec:org2131dd5}
\section*{Discretization}
\label{sec:orgfea45b0}
\subsection*{Domain}
\label{sec:org5ab43b0}
\begin{equation}
\Omega_d = \left\{ i,j \mid i,j \in \mathbb{N} \,, i,j \in [2,N+1] \right\}
\end{equation}
\begin{equation}
\begin{aligned}
\phi_{ij}^n: \Omega_d \times \left\{ 0, \dots  \right\} &\to \mathbb{R}\\
\mu_{ij}^n: \Omega_d \times \left\{ 0, \dots \right\} &\to \mathbb{R}
\end{aligned}
\end{equation}
\begin{align*}
G_{ij} &=
\begin{cases}
1, & i,j \in [2,N+1]  \\
0, & \text{else}
\end{cases}
\end{align*}
\subsection*{Finite Differences}
\label{sec:org89a6614}
\begin{align}
D_x\phi^{n+1,m}_{i+\frac{1}{2} j} &= \frac{\phi^{n+1,m}_{i+1j} - \phi^{n+1,m}_{ij}}{h} & D_y\phi^{n+1,m}_{ij+\frac{1}{2}} &= \frac{\phi^{n+1,m}_{ij+1} - \phi^{n+1,m}_{ij}}{h}
\end{align}
We define \(D_x\mu_{ij}^{n+\frac{1}{2},m} , D_y\mu_{ij}^{n+\frac{1}{2},m}\) in the same way.
\subsection*{Discrete CH equation}
\label{sec:orge3cb03d}
\begin{equation}
\label{eq:discrete-cahn-hilliard}
\begin{aligned}
\frac{\phi_{ij}^{n+1} - \phi_{ij}^n}{\Delta t}  &=  \nabla _d \cdot (G_{ij} \nabla_d \mu_{ij}^{n+\frac{1}{2}} )  \,, \\
 \mu_{ij}^{n+\frac{1}{2}} &= 2\phi_{ij}^{n+1} - \varepsilon^2  \nabla_d \cdot  (G_{ij} \nabla _d \phi_{ij}^{n+1} ) + W'(\phi_{ij}^n) - 2\phi _{ij}^n \,,
\end{aligned}
\end{equation}
\subsection*{Discrete CH equation}
\label{sec:org2083b10}
\begin{equation}
\label{eq:discrete-relaxed-cahn-hilliard}
\begin{aligned}
\frac{\phi_{ij}^{n+1,\alpha} - \phi_{ij}^{n,\alpha}}{\Delta t}  &=  \nabla _d \cdot (G_{ij} \nabla_d \mu_{ij}^{n+\frac{1}{2},\alpha} )  \,,\\
 \mu_{ij}^{n+\frac{1}{2},\alpha} &= 2\phi_{ij}^{n+1,\alpha} - \varepsilon^2 a(c_{ij}^{n+1,\alpha} - \phi_{ij}^{n+1,\alpha})  + W'(\phi_{ij}^{n,\alpha}) - 2\phi _{ij}^{n,\alpha} \,.
\end{aligned}
\end{equation}
\subsection*{b}
\label{sec:org8b0c089}
\begin{align*}
\begin{pmatrix}
\zeta^n_{ij}
 \\
\psi^n_{ij}
\end{pmatrix}
&=
\begin{pmatrix}
\frac{\phi_{ij}^{n}}{\Delta t}\\
W'(\phi_{ij}^n) - 2\phi_{ij}^n
\end{pmatrix}
.
\end{align*}
\end{document}
